% Options for packages loaded elsewhere
\PassOptionsToPackage{unicode}{hyperref}
\PassOptionsToPackage{hyphens}{url}
\PassOptionsToPackage{dvipsnames,svgnames,x11names}{xcolor}
%
\documentclass[
  a4paper,
  DIV=11,
  numbers=noendperiod]{scrartcl}

\usepackage{amsmath,amssymb}
\usepackage{iftex}
\ifPDFTeX
  \usepackage[T1]{fontenc}
  \usepackage[utf8]{inputenc}
  \usepackage{textcomp} % provide euro and other symbols
\else % if luatex or xetex
  \usepackage{unicode-math}
  \defaultfontfeatures{Scale=MatchLowercase}
  \defaultfontfeatures[\rmfamily]{Ligatures=TeX,Scale=1}
\fi
\usepackage{lmodern}
\ifPDFTeX\else  
    % xetex/luatex font selection
\fi
% Use upquote if available, for straight quotes in verbatim environments
\IfFileExists{upquote.sty}{\usepackage{upquote}}{}
\IfFileExists{microtype.sty}{% use microtype if available
  \usepackage[]{microtype}
  \UseMicrotypeSet[protrusion]{basicmath} % disable protrusion for tt fonts
}{}
\makeatletter
\@ifundefined{KOMAClassName}{% if non-KOMA class
  \IfFileExists{parskip.sty}{%
    \usepackage{parskip}
  }{% else
    \setlength{\parindent}{0pt}
    \setlength{\parskip}{6pt plus 2pt minus 1pt}}
}{% if KOMA class
  \KOMAoptions{parskip=half}}
\makeatother
\usepackage{xcolor}
\usepackage[top=20mm,bottom=20mm,left=20mm,right=20mm,heightrounded]{geometry}
\setlength{\emergencystretch}{3em} % prevent overfull lines
\setcounter{secnumdepth}{-\maxdimen} % remove section numbering
% Make \paragraph and \subparagraph free-standing
\ifx\paragraph\undefined\else
  \let\oldparagraph\paragraph
  \renewcommand{\paragraph}[1]{\oldparagraph{#1}\mbox{}}
\fi
\ifx\subparagraph\undefined\else
  \let\oldsubparagraph\subparagraph
  \renewcommand{\subparagraph}[1]{\oldsubparagraph{#1}\mbox{}}
\fi

\usepackage{color}
\usepackage{fancyvrb}
\newcommand{\VerbBar}{|}
\newcommand{\VERB}{\Verb[commandchars=\\\{\}]}
\DefineVerbatimEnvironment{Highlighting}{Verbatim}{commandchars=\\\{\}}
% Add ',fontsize=\small' for more characters per line
\usepackage{framed}
\definecolor{shadecolor}{RGB}{241,243,245}
\newenvironment{Shaded}{\begin{snugshade}}{\end{snugshade}}
\newcommand{\AlertTok}[1]{\textcolor[rgb]{0.68,0.00,0.00}{#1}}
\newcommand{\AnnotationTok}[1]{\textcolor[rgb]{0.37,0.37,0.37}{#1}}
\newcommand{\AttributeTok}[1]{\textcolor[rgb]{0.40,0.45,0.13}{#1}}
\newcommand{\BaseNTok}[1]{\textcolor[rgb]{0.68,0.00,0.00}{#1}}
\newcommand{\BuiltInTok}[1]{\textcolor[rgb]{0.00,0.23,0.31}{#1}}
\newcommand{\CharTok}[1]{\textcolor[rgb]{0.13,0.47,0.30}{#1}}
\newcommand{\CommentTok}[1]{\textcolor[rgb]{0.37,0.37,0.37}{#1}}
\newcommand{\CommentVarTok}[1]{\textcolor[rgb]{0.37,0.37,0.37}{\textit{#1}}}
\newcommand{\ConstantTok}[1]{\textcolor[rgb]{0.56,0.35,0.01}{#1}}
\newcommand{\ControlFlowTok}[1]{\textcolor[rgb]{0.00,0.23,0.31}{#1}}
\newcommand{\DataTypeTok}[1]{\textcolor[rgb]{0.68,0.00,0.00}{#1}}
\newcommand{\DecValTok}[1]{\textcolor[rgb]{0.68,0.00,0.00}{#1}}
\newcommand{\DocumentationTok}[1]{\textcolor[rgb]{0.37,0.37,0.37}{\textit{#1}}}
\newcommand{\ErrorTok}[1]{\textcolor[rgb]{0.68,0.00,0.00}{#1}}
\newcommand{\ExtensionTok}[1]{\textcolor[rgb]{0.00,0.23,0.31}{#1}}
\newcommand{\FloatTok}[1]{\textcolor[rgb]{0.68,0.00,0.00}{#1}}
\newcommand{\FunctionTok}[1]{\textcolor[rgb]{0.28,0.35,0.67}{#1}}
\newcommand{\ImportTok}[1]{\textcolor[rgb]{0.00,0.46,0.62}{#1}}
\newcommand{\InformationTok}[1]{\textcolor[rgb]{0.37,0.37,0.37}{#1}}
\newcommand{\KeywordTok}[1]{\textcolor[rgb]{0.00,0.23,0.31}{#1}}
\newcommand{\NormalTok}[1]{\textcolor[rgb]{0.00,0.23,0.31}{#1}}
\newcommand{\OperatorTok}[1]{\textcolor[rgb]{0.37,0.37,0.37}{#1}}
\newcommand{\OtherTok}[1]{\textcolor[rgb]{0.00,0.23,0.31}{#1}}
\newcommand{\PreprocessorTok}[1]{\textcolor[rgb]{0.68,0.00,0.00}{#1}}
\newcommand{\RegionMarkerTok}[1]{\textcolor[rgb]{0.00,0.23,0.31}{#1}}
\newcommand{\SpecialCharTok}[1]{\textcolor[rgb]{0.37,0.37,0.37}{#1}}
\newcommand{\SpecialStringTok}[1]{\textcolor[rgb]{0.13,0.47,0.30}{#1}}
\newcommand{\StringTok}[1]{\textcolor[rgb]{0.13,0.47,0.30}{#1}}
\newcommand{\VariableTok}[1]{\textcolor[rgb]{0.07,0.07,0.07}{#1}}
\newcommand{\VerbatimStringTok}[1]{\textcolor[rgb]{0.13,0.47,0.30}{#1}}
\newcommand{\WarningTok}[1]{\textcolor[rgb]{0.37,0.37,0.37}{\textit{#1}}}

\providecommand{\tightlist}{%
  \setlength{\itemsep}{0pt}\setlength{\parskip}{0pt}}\usepackage{longtable,booktabs,array}
\usepackage{calc} % for calculating minipage widths
% Correct order of tables after \paragraph or \subparagraph
\usepackage{etoolbox}
\makeatletter
\patchcmd\longtable{\par}{\if@noskipsec\mbox{}\fi\par}{}{}
\makeatother
% Allow footnotes in longtable head/foot
\IfFileExists{footnotehyper.sty}{\usepackage{footnotehyper}}{\usepackage{footnote}}
\makesavenoteenv{longtable}
\usepackage{graphicx}
\makeatletter
\def\maxwidth{\ifdim\Gin@nat@width>\linewidth\linewidth\else\Gin@nat@width\fi}
\def\maxheight{\ifdim\Gin@nat@height>\textheight\textheight\else\Gin@nat@height\fi}
\makeatother
% Scale images if necessary, so that they will not overflow the page
% margins by default, and it is still possible to overwrite the defaults
% using explicit options in \includegraphics[width, height, ...]{}
\setkeys{Gin}{width=\maxwidth,height=\maxheight,keepaspectratio}
% Set default figure placement to htbp
\makeatletter
\def\fps@figure{htbp}
\makeatother

\usepackage{fancyhdr} \pagestyle{fancy} \usepackage{lastpage}
\KOMAoption{captions}{tablesignature}
\makeatletter
\@ifpackageloaded{tcolorbox}{}{\usepackage[skins,breakable]{tcolorbox}}
\@ifpackageloaded{fontawesome5}{}{\usepackage{fontawesome5}}
\definecolor{quarto-callout-color}{HTML}{909090}
\definecolor{quarto-callout-note-color}{HTML}{0758E5}
\definecolor{quarto-callout-important-color}{HTML}{CC1914}
\definecolor{quarto-callout-warning-color}{HTML}{EB9113}
\definecolor{quarto-callout-tip-color}{HTML}{00A047}
\definecolor{quarto-callout-caution-color}{HTML}{FC5300}
\definecolor{quarto-callout-color-frame}{HTML}{acacac}
\definecolor{quarto-callout-note-color-frame}{HTML}{4582ec}
\definecolor{quarto-callout-important-color-frame}{HTML}{d9534f}
\definecolor{quarto-callout-warning-color-frame}{HTML}{f0ad4e}
\definecolor{quarto-callout-tip-color-frame}{HTML}{02b875}
\definecolor{quarto-callout-caution-color-frame}{HTML}{fd7e14}
\makeatother
\makeatletter
\makeatother
\makeatletter
\makeatother
\makeatletter
\@ifpackageloaded{caption}{}{\usepackage{caption}}
\AtBeginDocument{%
\ifdefined\contentsname
  \renewcommand*\contentsname{Table des matières}
\else
  \newcommand\contentsname{Table des matières}
\fi
\ifdefined\listfigurename
  \renewcommand*\listfigurename{Liste des Figures}
\else
  \newcommand\listfigurename{Liste des Figures}
\fi
\ifdefined\listtablename
  \renewcommand*\listtablename{Liste des Tables}
\else
  \newcommand\listtablename{Liste des Tables}
\fi
\ifdefined\figurename
  \renewcommand*\figurename{Figure}
\else
  \newcommand\figurename{Figure}
\fi
\ifdefined\tablename
  \renewcommand*\tablename{Tableau}
\else
  \newcommand\tablename{Tableau}
\fi
}
\@ifpackageloaded{float}{}{\usepackage{float}}
\floatstyle{ruled}
\@ifundefined{c@chapter}{\newfloat{codelisting}{h}{lop}}{\newfloat{codelisting}{h}{lop}[chapter]}
\floatname{codelisting}{Listing}
\newcommand*\listoflistings{\listof{codelisting}{Liste des Listings}}
\makeatother
\makeatletter
\@ifpackageloaded{caption}{}{\usepackage{caption}}
\@ifpackageloaded{subcaption}{}{\usepackage{subcaption}}
\makeatother
\makeatletter
\@ifpackageloaded{tcolorbox}{}{\usepackage[skins,breakable]{tcolorbox}}
\makeatother
\makeatletter
\@ifundefined{shadecolor}{\definecolor{shadecolor}{rgb}{.97, .97, .97}}
\makeatother
\makeatletter
\makeatother
\makeatletter
\makeatother
\makeatletter
\@ifpackageloaded{tikz}{}{\usepackage{tikz}}
\makeatother
        \newcommand*\circled[1]{\tikz[baseline=(char.base)]{
          \node[shape=circle,draw,inner sep=1pt] (char) {{\scriptsize#1}};}}  
                  
\ifLuaTeX
\usepackage[bidi=basic]{babel}
\else
\usepackage[bidi=default]{babel}
\fi
\babelprovide[main,import]{french}
% get rid of language-specific shorthands (see #6817):
\let\LanguageShortHands\languageshorthands
\def\languageshorthands#1{}
\ifLuaTeX
  \usepackage{selnolig}  % disable illegal ligatures
\fi
\IfFileExists{bookmark.sty}{\usepackage{bookmark}}{\usepackage{hyperref}}
\IfFileExists{xurl.sty}{\usepackage{xurl}}{} % add URL line breaks if available
\urlstyle{same} % disable monospaced font for URLs
\hypersetup{
  pdftitle={Algorithmes de tris},
  pdflang={fr},
  colorlinks=true,
  linkcolor={blue},
  filecolor={Maroon},
  citecolor={Blue},
  urlcolor={Blue},
  pdfcreator={LaTeX via pandoc}}

\title{Algorithmes de tris}
\usepackage{etoolbox}
\makeatletter
\providecommand{\subtitle}[1]{% add subtitle to \maketitle
  \apptocmd{\@title}{\par {\large #1 \par}}{}{}
}
\makeatother
\subtitle{S6 - Algorithmique (1)}
\author{}
\date{}

\begin{document}
\maketitle
\lhead{Spécialité NSI} \rhead{Première} \chead{} \cfoot{} \lfoot{Lycée \'Emile Duclaux} \rfoot{Page \thepage/\pageref{LastPage}} \renewcommand{\headrulewidth}{0pt} \renewcommand{\footrulewidth}{0pt} \thispagestyle{fancy} \vspace{-2cm}

\ifdefined\Shaded\renewenvironment{Shaded}{\begin{tcolorbox}[enhanced, interior hidden, sharp corners, boxrule=0pt, frame hidden, borderline west={3pt}{0pt}{shadecolor}, breakable]}{\end{tcolorbox}}\fi

Dans cette partie du cours, nous allons étudier deux algorithmes de tris
: le tri par insertion et le tri par sélection.

Étant donné un tableau de nombres, l'objectif est d'écrire une fonction
qui renvoie un tableau contenant les mêmes nombres mais dans l'ordre
croissant.

\hypertarget{tri-par-insertion}{%
\subsection{1. Tri par insertion}\label{tri-par-insertion}}

\hypertarget{le-principe}{%
\subsubsection{Le principe}\label{le-principe}}

\begin{tcolorbox}[enhanced jigsaw, titlerule=0mm, toptitle=1mm, bottomtitle=1mm, opacitybacktitle=0.6, left=2mm, breakable, opacityback=0, toprule=.15mm, colback=white, title=\textcolor{quarto-callout-important-color}{\faExclamation}\hspace{0.5em}{Principe de l'algorithme}, colbacktitle=quarto-callout-important-color!10!white, coltitle=black, colframe=quarto-callout-important-color-frame, bottomrule=.15mm, arc=.35mm, rightrule=.15mm, leftrule=.75mm]

En commençant par le deuxième élément du tableau :

\begin{itemize}
\tightlist
\item
  On compare l'élément courant avec l'élément précédent.
\item
  Si l'élément courant est plus petit, on échange les deux éléments.
\item
  On continue à comparer et échanger l'élément courant avec les éléments
  précédents jusqu'à ce que l'élément courant soit plus grand que
  l'élément précédent.
\end{itemize}

\end{tcolorbox}

Animation du tri par insertion :

\hypertarget{insertion}{}

\hypertarget{programmation}{%
\subsubsection{Programmation}\label{programmation}}

\hypertarget{annotated-cell-1}{%
\label{annotated-cell-1}}%
\begin{Shaded}
\begin{Highlighting}[]
\KeywordTok{def}\NormalTok{ tri\_insertion(tableau: }\BuiltInTok{list}\NormalTok{) }\OperatorTok{{-}\textgreater{}} \BuiltInTok{list}\NormalTok{:}
    \CommentTok{"""Tri en place par insertion le tableau passé en paramètre."""}
    \ControlFlowTok{for}\NormalTok{ i }\KeywordTok{in} \BuiltInTok{range}\NormalTok{(}\DecValTok{1}\NormalTok{, }\BuiltInTok{len}\NormalTok{(tableau)):}\hspace*{\fill}\NormalTok{\circled{1}}
\NormalTok{        j }\OperatorTok{=}\NormalTok{ i}\hspace*{\fill}\NormalTok{\circled{2}}
        \ControlFlowTok{while}\NormalTok{ j }\OperatorTok{\textgreater{}} \DecValTok{0} \KeywordTok{and}\NormalTok{ tableau[j] }\OperatorTok{\textless{}}\NormalTok{ tableau[j}\OperatorTok{{-}}\DecValTok{1}\NormalTok{]:}\hspace*{\fill}\NormalTok{\circled{3}}
\NormalTok{            tableau[j], tableau[j}\OperatorTok{{-}}\DecValTok{1}\NormalTok{] }\OperatorTok{=}\NormalTok{ tableau[j}\OperatorTok{{-}}\DecValTok{1}\NormalTok{], tableau[j]}\hspace*{\fill}\NormalTok{\circled{4}}
\NormalTok{            j }\OperatorTok{{-}=} \DecValTok{1}\hspace*{\fill}\NormalTok{\circled{5}}
    \ControlFlowTok{return}\NormalTok{ tableau}
\end{Highlighting}
\end{Shaded}

\begin{description}
\tightlist
\item[\circled{1}]
On commence à l'indice 1 qui correspond au deuxième élément du tableau.
\item[\circled{2}]
On stocke l'indice courant dans une variable \texttt{j} pour pouvoir le
modifier.
\item[\circled{3}]
Tant que l'indice courant est supérieur à 0 et que l'élément courant est
plus petit que l'élément précédent, on échange les deux éléments.
\item[\circled{4}]
On échange les deux éléments.
\item[\circled{5}]
L'élément courant est maintenant l'élément précédent, on décrémente donc
l'indice courant.
\end{description}

Test de l'algorithme :

\begin{Shaded}
\begin{Highlighting}[]
\NormalTok{tri\_insertion([}\DecValTok{5}\NormalTok{, }\DecValTok{2}\NormalTok{, }\DecValTok{4}\NormalTok{, }\DecValTok{6}\NormalTok{, }\DecValTok{1}\NormalTok{, }\DecValTok{3}\NormalTok{])}
\end{Highlighting}
\end{Shaded}

\begin{verbatim}
[1, 2, 3, 4, 5, 6]
\end{verbatim}

\hypertarget{preuve-de-terminaison}{%
\subsubsection{Preuve de terminaison}\label{preuve-de-terminaison}}

Montrons que l'algorithme termine.

D'une part, il est certain que la boucle \texttt{for}, boucle bornée par
nature, se termine. D'autre part, la boucle \texttt{while} se termine
aussi. La variable \texttt{j} peut être est un \textbf{variant de
boucle}. À chaque itération, sa valeur de diminue de 1 : elle finit donc
toujours par atteindre 0.

La terminaison de l'algorithme est donc prouvée.

\hypertarget{preuve-de-correction}{%
\subsubsection{Preuve de correction}\label{preuve-de-correction}}

Montrons que l'algorithme trie bien le tableau.

Pour cela, considérons la propriété suivante : à chaque itération, le
sous-tableau composé des \texttt{i} premiers éléments est trié. Montrons
que cette propriété est un \textbf{invariant de boucle}.

\begin{itemize}
\tightlist
\item
  \textbf{Initialisation} : au début de l'algorithme, le sous-tableau
  composé uniquement du premier élément est trié.
\item
  \textbf{Conservation} : supposons que le le sous-tableau composé des
  \texttt{i} premiers éléments est trié :
  \([e_0, e_1, \ldots, e_{i-1}]\) avec
  \(e_0\leqslant e_1\leqslant \ldots \leqslant e_{i-1}\). L'algorithme
  considère alors l'élément \(e_i\) et le compare avec les éléments
  précédents. Si \(e_i\) est plus petit que \(e_{i-1}\), on échange les
  deux éléments. On continue alors à comparer \(e_i\) avec les éléments
  précédents jusqu'à ce que \(e_i\) soit plus grand que l'élément
  précédent. Le sous-tableau composé des \texttt{i+1} premiers éléments
  est alors trié.
\item
  \textbf{Termination} : à la fin de l'algorithme \texttt{i} a la valeur
  \texttt{n-1} ce qui correspond à l'indice du dernier élément du
  tableau. Le sous-tableau composé des \texttt{n} premiers éléments est
  donc trié. Or, \texttt{n} est le nombre d'éléments du tableau, donc le
  tableau entier est trié.
\end{itemize}

La correction de l'algorithme est donc prouvée.

\hypertarget{complexituxe9}{%
\subsubsection{Complexité}\label{complexituxe9}}

\hypertarget{tri-par-suxe9lection}{%
\subsection{2. Tri par sélection}\label{tri-par-suxe9lection}}

\hypertarget{selection}{}



\end{document}
