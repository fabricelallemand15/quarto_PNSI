% Options for packages loaded elsewhere
\PassOptionsToPackage{unicode}{hyperref}
\PassOptionsToPackage{hyphens}{url}
\PassOptionsToPackage{dvipsnames,svgnames,x11names}{xcolor}
%
\documentclass[
  a4paper,
  DIV=11,
  numbers=noendperiod]{scrartcl}

\usepackage{amsmath,amssymb}
\usepackage{iftex}
\ifPDFTeX
  \usepackage[T1]{fontenc}
  \usepackage[utf8]{inputenc}
  \usepackage{textcomp} % provide euro and other symbols
\else % if luatex or xetex
  \usepackage{unicode-math}
  \defaultfontfeatures{Scale=MatchLowercase}
  \defaultfontfeatures[\rmfamily]{Ligatures=TeX,Scale=1}
\fi
\usepackage{lmodern}
\ifPDFTeX\else  
    % xetex/luatex font selection
\fi
% Use upquote if available, for straight quotes in verbatim environments
\IfFileExists{upquote.sty}{\usepackage{upquote}}{}
\IfFileExists{microtype.sty}{% use microtype if available
  \usepackage[]{microtype}
  \UseMicrotypeSet[protrusion]{basicmath} % disable protrusion for tt fonts
}{}
\makeatletter
\@ifundefined{KOMAClassName}{% if non-KOMA class
  \IfFileExists{parskip.sty}{%
    \usepackage{parskip}
  }{% else
    \setlength{\parindent}{0pt}
    \setlength{\parskip}{6pt plus 2pt minus 1pt}}
}{% if KOMA class
  \KOMAoptions{parskip=half}}
\makeatother
\usepackage{xcolor}
\usepackage[top=20mm,bottom=20mm,left=20mm,right=20mm,heightrounded]{geometry}
\setlength{\emergencystretch}{3em} % prevent overfull lines
\setcounter{secnumdepth}{-\maxdimen} % remove section numbering
% Make \paragraph and \subparagraph free-standing
\ifx\paragraph\undefined\else
  \let\oldparagraph\paragraph
  \renewcommand{\paragraph}[1]{\oldparagraph{#1}\mbox{}}
\fi
\ifx\subparagraph\undefined\else
  \let\oldsubparagraph\subparagraph
  \renewcommand{\subparagraph}[1]{\oldsubparagraph{#1}\mbox{}}
\fi

\usepackage{color}
\usepackage{fancyvrb}
\newcommand{\VerbBar}{|}
\newcommand{\VERB}{\Verb[commandchars=\\\{\}]}
\DefineVerbatimEnvironment{Highlighting}{Verbatim}{commandchars=\\\{\}}
% Add ',fontsize=\small' for more characters per line
\usepackage{framed}
\definecolor{shadecolor}{RGB}{241,243,245}
\newenvironment{Shaded}{\begin{snugshade}}{\end{snugshade}}
\newcommand{\AlertTok}[1]{\textcolor[rgb]{0.68,0.00,0.00}{#1}}
\newcommand{\AnnotationTok}[1]{\textcolor[rgb]{0.37,0.37,0.37}{#1}}
\newcommand{\AttributeTok}[1]{\textcolor[rgb]{0.40,0.45,0.13}{#1}}
\newcommand{\BaseNTok}[1]{\textcolor[rgb]{0.68,0.00,0.00}{#1}}
\newcommand{\BuiltInTok}[1]{\textcolor[rgb]{0.00,0.23,0.31}{#1}}
\newcommand{\CharTok}[1]{\textcolor[rgb]{0.13,0.47,0.30}{#1}}
\newcommand{\CommentTok}[1]{\textcolor[rgb]{0.37,0.37,0.37}{#1}}
\newcommand{\CommentVarTok}[1]{\textcolor[rgb]{0.37,0.37,0.37}{\textit{#1}}}
\newcommand{\ConstantTok}[1]{\textcolor[rgb]{0.56,0.35,0.01}{#1}}
\newcommand{\ControlFlowTok}[1]{\textcolor[rgb]{0.00,0.23,0.31}{#1}}
\newcommand{\DataTypeTok}[1]{\textcolor[rgb]{0.68,0.00,0.00}{#1}}
\newcommand{\DecValTok}[1]{\textcolor[rgb]{0.68,0.00,0.00}{#1}}
\newcommand{\DocumentationTok}[1]{\textcolor[rgb]{0.37,0.37,0.37}{\textit{#1}}}
\newcommand{\ErrorTok}[1]{\textcolor[rgb]{0.68,0.00,0.00}{#1}}
\newcommand{\ExtensionTok}[1]{\textcolor[rgb]{0.00,0.23,0.31}{#1}}
\newcommand{\FloatTok}[1]{\textcolor[rgb]{0.68,0.00,0.00}{#1}}
\newcommand{\FunctionTok}[1]{\textcolor[rgb]{0.28,0.35,0.67}{#1}}
\newcommand{\ImportTok}[1]{\textcolor[rgb]{0.00,0.46,0.62}{#1}}
\newcommand{\InformationTok}[1]{\textcolor[rgb]{0.37,0.37,0.37}{#1}}
\newcommand{\KeywordTok}[1]{\textcolor[rgb]{0.00,0.23,0.31}{#1}}
\newcommand{\NormalTok}[1]{\textcolor[rgb]{0.00,0.23,0.31}{#1}}
\newcommand{\OperatorTok}[1]{\textcolor[rgb]{0.37,0.37,0.37}{#1}}
\newcommand{\OtherTok}[1]{\textcolor[rgb]{0.00,0.23,0.31}{#1}}
\newcommand{\PreprocessorTok}[1]{\textcolor[rgb]{0.68,0.00,0.00}{#1}}
\newcommand{\RegionMarkerTok}[1]{\textcolor[rgb]{0.00,0.23,0.31}{#1}}
\newcommand{\SpecialCharTok}[1]{\textcolor[rgb]{0.37,0.37,0.37}{#1}}
\newcommand{\SpecialStringTok}[1]{\textcolor[rgb]{0.13,0.47,0.30}{#1}}
\newcommand{\StringTok}[1]{\textcolor[rgb]{0.13,0.47,0.30}{#1}}
\newcommand{\VariableTok}[1]{\textcolor[rgb]{0.07,0.07,0.07}{#1}}
\newcommand{\VerbatimStringTok}[1]{\textcolor[rgb]{0.13,0.47,0.30}{#1}}
\newcommand{\WarningTok}[1]{\textcolor[rgb]{0.37,0.37,0.37}{\textit{#1}}}

\providecommand{\tightlist}{%
  \setlength{\itemsep}{0pt}\setlength{\parskip}{0pt}}\usepackage{longtable,booktabs,array}
\usepackage{calc} % for calculating minipage widths
% Correct order of tables after \paragraph or \subparagraph
\usepackage{etoolbox}
\makeatletter
\patchcmd\longtable{\par}{\if@noskipsec\mbox{}\fi\par}{}{}
\makeatother
% Allow footnotes in longtable head/foot
\IfFileExists{footnotehyper.sty}{\usepackage{footnotehyper}}{\usepackage{footnote}}
\makesavenoteenv{longtable}
\usepackage{graphicx}
\makeatletter
\def\maxwidth{\ifdim\Gin@nat@width>\linewidth\linewidth\else\Gin@nat@width\fi}
\def\maxheight{\ifdim\Gin@nat@height>\textheight\textheight\else\Gin@nat@height\fi}
\makeatother
% Scale images if necessary, so that they will not overflow the page
% margins by default, and it is still possible to overwrite the defaults
% using explicit options in \includegraphics[width, height, ...]{}
\setkeys{Gin}{width=\maxwidth,height=\maxheight,keepaspectratio}
% Set default figure placement to htbp
\makeatletter
\def\fps@figure{htbp}
\makeatother

\usepackage{fancyhdr} \pagestyle{fancy} \usepackage{lastpage}
\KOMAoption{captions}{tablesignature}
\makeatletter
\makeatother
\makeatletter
\makeatother
\makeatletter
\@ifpackageloaded{caption}{}{\usepackage{caption}}
\AtBeginDocument{%
\ifdefined\contentsname
  \renewcommand*\contentsname{Table des matières}
\else
  \newcommand\contentsname{Table des matières}
\fi
\ifdefined\listfigurename
  \renewcommand*\listfigurename{Liste des Figures}
\else
  \newcommand\listfigurename{Liste des Figures}
\fi
\ifdefined\listtablename
  \renewcommand*\listtablename{Liste des Tables}
\else
  \newcommand\listtablename{Liste des Tables}
\fi
\ifdefined\figurename
  \renewcommand*\figurename{Figure}
\else
  \newcommand\figurename{Figure}
\fi
\ifdefined\tablename
  \renewcommand*\tablename{Tableau}
\else
  \newcommand\tablename{Tableau}
\fi
}
\@ifpackageloaded{float}{}{\usepackage{float}}
\floatstyle{ruled}
\@ifundefined{c@chapter}{\newfloat{codelisting}{h}{lop}}{\newfloat{codelisting}{h}{lop}[chapter]}
\floatname{codelisting}{Listing}
\newcommand*\listoflistings{\listof{codelisting}{Liste des Listings}}
\makeatother
\makeatletter
\@ifpackageloaded{caption}{}{\usepackage{caption}}
\@ifpackageloaded{subcaption}{}{\usepackage{subcaption}}
\makeatother
\makeatletter
\@ifpackageloaded{tcolorbox}{}{\usepackage[skins,breakable]{tcolorbox}}
\makeatother
\makeatletter
\@ifundefined{shadecolor}{\definecolor{shadecolor}{rgb}{.97, .97, .97}}
\makeatother
\makeatletter
\makeatother
\makeatletter
\makeatother
\ifLuaTeX
\usepackage[bidi=basic]{babel}
\else
\usepackage[bidi=default]{babel}
\fi
\babelprovide[main,import]{french}
% get rid of language-specific shorthands (see #6817):
\let\LanguageShortHands\languageshorthands
\def\languageshorthands#1{}
\ifLuaTeX
  \usepackage{selnolig}  % disable illegal ligatures
\fi
\IfFileExists{bookmark.sty}{\usepackage{bookmark}}{\usepackage{hyperref}}
\IfFileExists{xurl.sty}{\usepackage{xurl}}{} % add URL line breaks if available
\urlstyle{same} % disable monospaced font for URLs
\hypersetup{
  pdftitle={Parcours séquentiel d'un tableau},
  pdflang={fr},
  colorlinks=true,
  linkcolor={blue},
  filecolor={Maroon},
  citecolor={Blue},
  urlcolor={Blue},
  pdfcreator={LaTeX via pandoc}}

\title{Parcours séquentiel d'un tableau}
\usepackage{etoolbox}
\makeatletter
\providecommand{\subtitle}[1]{% add subtitle to \maketitle
  \apptocmd{\@title}{\par {\large #1 \par}}{}{}
}
\makeatother
\subtitle{S6 - Algorithmique (1)}
\author{}
\date{}

\begin{document}
\maketitle
\lhead{Spécialité NSI} \rhead{Première} \chead{} \cfoot{} \lfoot{Lycée \'Emile Duclaux} \rfoot{Page \thepage/\pageref{LastPage}} \renewcommand{\headrulewidth}{0pt} \renewcommand{\footrulewidth}{0pt} \thispagestyle{fancy} \vspace{-2cm}

\ifdefined\Shaded\renewenvironment{Shaded}{\begin{tcolorbox}[frame hidden, boxrule=0pt, interior hidden, breakable, borderline west={3pt}{0pt}{shadecolor}, enhanced, sharp corners]}{\end{tcolorbox}}\fi

\hypertarget{recherche-dune-occurrence}{%
\subsection{1. Recherche d'une
occurrence}\label{recherche-dune-occurrence}}

Considérons un tableau. On souhaite disposer d'un algorithme permettant
de rechercher une occurrence d'une valeur donnée dans ce tableau. Plus
précisément, nous allons définir une fonction qui recherche une valeur
donnée dans un tableau et qui retourne le tableau des indices des
occurrences de cette valeur dans le tableau. Dans le cas où la valeur
n'est pas présente dans le tableau, la fonction retournera un tableau
vide.

La méthode est très simple. On parcourt le tableau en testant à chaque
fois si la valeur courante est égale à la valeur recherchée. Si c'est le
cas, on ajoute l'indice de la valeur courante dans le tableau des
indices des occurrences.

\begin{Shaded}
\begin{Highlighting}[]
\KeywordTok{def}\NormalTok{ occurrences(tab, val):}
    \CommentTok{"""Retourne un tableau contenant les indices des occurrences de val dans tab"""}
\NormalTok{    indices }\OperatorTok{=}\NormalTok{ []}
    \ControlFlowTok{for}\NormalTok{ i }\KeywordTok{in} \BuiltInTok{range}\NormalTok{(}\BuiltInTok{len}\NormalTok{(tab)):}
        \ControlFlowTok{if}\NormalTok{ tab[i] }\OperatorTok{==}\NormalTok{ val:}
\NormalTok{            indices.append(i)}
    \ControlFlowTok{return}\NormalTok{ indices}
\end{Highlighting}
\end{Shaded}

Exemple d'utilisation :

\begin{Shaded}
\begin{Highlighting}[]
\NormalTok{tab }\OperatorTok{=}\NormalTok{ [}\StringTok{"DO"}\NormalTok{, }\StringTok{"RE"}\NormalTok{, }\StringTok{"MI"}\NormalTok{, }\StringTok{"FA"}\NormalTok{, }\StringTok{"SOL"}\NormalTok{, }\StringTok{"LA"}\NormalTok{, }\StringTok{"SI"}\NormalTok{, }\StringTok{"DO"}\NormalTok{]}
\BuiltInTok{print}\NormalTok{(occurrences(tab, }\StringTok{"DO"}\NormalTok{))}
\BuiltInTok{print}\NormalTok{(occurrences(tab, }\StringTok{"MI"}\NormalTok{))}
\BuiltInTok{print}\NormalTok{(occurrences(tab, }\StringTok{"UT"}\NormalTok{))}
\end{Highlighting}
\end{Shaded}

\begin{verbatim}
[0, 7]
[2]
[]
\end{verbatim}

\textbf{Complexité de l'algorithme} : Comptons le nombre d'itérations et
de tests. Notons \(n\) la taille du tableau, l'algorithme parcourt
toutes les valeurs du tableau. Il y a donc au total \(n\) itérations. De
plus, nous avons \(n\) tests. Il y a donc \(2n\) opérations au total. On
peut donc dire que l'algorithme est de complexité \(\mathcal{O}(n)\).

\hypertarget{recherche-dun-extremum}{%
\subsection{2. Recherche d'un extremum}\label{recherche-dun-extremum}}

Dans cette partie, nous considérons un tableau dont les éléments sont
des nombres. Nous allons définir une fonction qui recherche le plus
grand élément du tableau. Pour cela, on commence par choisir comme
maximum temporaire le premier élément du tableau. On parcourt ensuite le
tableau en testant à chaque fois si la valeur courante est plus grande
que le maximum temporaire. Si c'est le cas, on met à jour le maximum
temporaire avec la valeur courante.

\begin{Shaded}
\begin{Highlighting}[]
\KeywordTok{def} \BuiltInTok{max}\NormalTok{(tab):}
    \CommentTok{"""Retourne le plus grand élément du tableau"""}
\NormalTok{    m }\OperatorTok{=}\NormalTok{ tab[}\DecValTok{0}\NormalTok{]}
    \ControlFlowTok{for}\NormalTok{ i }\KeywordTok{in} \BuiltInTok{range}\NormalTok{(}\DecValTok{1}\NormalTok{, }\BuiltInTok{len}\NormalTok{(tab)):}
        \ControlFlowTok{if}\NormalTok{ tab[i] }\OperatorTok{\textgreater{}}\NormalTok{ m:}
\NormalTok{            m }\OperatorTok{=}\NormalTok{ tab[i]}
    \ControlFlowTok{return}\NormalTok{ m}
\end{Highlighting}
\end{Shaded}

Exemple d'utilisation :

\begin{Shaded}
\begin{Highlighting}[]
\NormalTok{tab }\OperatorTok{=}\NormalTok{ [}\DecValTok{201}\NormalTok{, }\DecValTok{203}\NormalTok{, }\DecValTok{35}\NormalTok{, }\DecValTok{448}\NormalTok{, }\DecValTok{55}\NormalTok{, }\DecValTok{16}\NormalTok{, }\DecValTok{2023}\NormalTok{, }\DecValTok{14}\NormalTok{, }\DecValTok{999}\NormalTok{, }\DecValTok{100}\NormalTok{]}
\BuiltInTok{print}\NormalTok{(}\BuiltInTok{max}\NormalTok{(tab))}
\end{Highlighting}
\end{Shaded}

\begin{verbatim}
2023
\end{verbatim}

\textbf{Complexité de l'algorithme} : La boucle \texttt{for} parcourt
toutes les valeurs du tableau, sauf la première. Il y a donc au total
\(n-1\) itérations. Nous avons aussi \(n-1\) comparaisons. Au total, le
nombre d'opérations est donc de \(2n-2\). On peut donc dire que
l'algorithme est de complexité \(\mathcal{O}(n)\).

\hypertarget{calcul-dune-moyenne}{%
\subsection{3. Calcul d'une moyenne}\label{calcul-dune-moyenne}}

Dans cette partie, nous considérons un tableau dont les éléments sont
des nombres. Nous allons définir une fonction qui calcule la moyenne des
éléments du tableau.

\begin{Shaded}
\begin{Highlighting}[]
\KeywordTok{def}\NormalTok{ moyenne(tab):}
    \CommentTok{"""Retourne la moyenne des éléments du tableau"""}
\NormalTok{    s }\OperatorTok{=} \DecValTok{0}
    \ControlFlowTok{for}\NormalTok{ i }\KeywordTok{in} \BuiltInTok{range}\NormalTok{(}\BuiltInTok{len}\NormalTok{(tab)):}
\NormalTok{        s }\OperatorTok{+=}\NormalTok{ tab[i]}
    \ControlFlowTok{return}\NormalTok{ s }\OperatorTok{/} \BuiltInTok{len}\NormalTok{(tab)}
\end{Highlighting}
\end{Shaded}

Exemple d'utilisation :

\begin{Shaded}
\begin{Highlighting}[]
\NormalTok{tab }\OperatorTok{=}\NormalTok{ [}\DecValTok{201}\NormalTok{, }\DecValTok{203}\NormalTok{, }\DecValTok{35}\NormalTok{, }\DecValTok{448}\NormalTok{, }\DecValTok{55}\NormalTok{, }\DecValTok{16}\NormalTok{, }\DecValTok{2023}\NormalTok{, }\DecValTok{14}\NormalTok{, }\DecValTok{999}\NormalTok{, }\DecValTok{100}\NormalTok{]}
\BuiltInTok{print}\NormalTok{(moyenne(tab))}
\end{Highlighting}
\end{Shaded}

\begin{verbatim}
409.4
\end{verbatim}

\textbf{Complexité de l'algorithme} : La boucle \texttt{for} parcourt
toutes les valeurs du tableau. Il y a donc au total \(n\) itérations.
Nous avons aussi \(n\) additions et une division. Au total, le nombre
d'opérations est donc de \(2n+1\). On peut donc dire que l'algorithme
est encore de complexité \(\mathcal{O}(n)\).



\end{document}
