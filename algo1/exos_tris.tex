% Options for packages loaded elsewhere
\PassOptionsToPackage{unicode}{hyperref}
\PassOptionsToPackage{hyphens}{url}
\PassOptionsToPackage{dvipsnames,svgnames,x11names}{xcolor}
%
\documentclass[
  a4paper,
  DIV=11,
  numbers=noendperiod]{scrartcl}

\usepackage{amsmath,amssymb}
\usepackage{iftex}
\ifPDFTeX
  \usepackage[T1]{fontenc}
  \usepackage[utf8]{inputenc}
  \usepackage{textcomp} % provide euro and other symbols
\else % if luatex or xetex
  \usepackage{unicode-math}
  \defaultfontfeatures{Scale=MatchLowercase}
  \defaultfontfeatures[\rmfamily]{Ligatures=TeX,Scale=1}
\fi
\usepackage{lmodern}
\ifPDFTeX\else  
    % xetex/luatex font selection
\fi
% Use upquote if available, for straight quotes in verbatim environments
\IfFileExists{upquote.sty}{\usepackage{upquote}}{}
\IfFileExists{microtype.sty}{% use microtype if available
  \usepackage[]{microtype}
  \UseMicrotypeSet[protrusion]{basicmath} % disable protrusion for tt fonts
}{}
\makeatletter
\@ifundefined{KOMAClassName}{% if non-KOMA class
  \IfFileExists{parskip.sty}{%
    \usepackage{parskip}
  }{% else
    \setlength{\parindent}{0pt}
    \setlength{\parskip}{6pt plus 2pt minus 1pt}}
}{% if KOMA class
  \KOMAoptions{parskip=half}}
\makeatother
\usepackage{xcolor}
\usepackage[top=20mm,bottom=20mm,left=20mm,right=20mm,heightrounded]{geometry}
\setlength{\emergencystretch}{3em} % prevent overfull lines
\setcounter{secnumdepth}{-\maxdimen} % remove section numbering
% Make \paragraph and \subparagraph free-standing
\ifx\paragraph\undefined\else
  \let\oldparagraph\paragraph
  \renewcommand{\paragraph}[1]{\oldparagraph{#1}\mbox{}}
\fi
\ifx\subparagraph\undefined\else
  \let\oldsubparagraph\subparagraph
  \renewcommand{\subparagraph}[1]{\oldsubparagraph{#1}\mbox{}}
\fi


\providecommand{\tightlist}{%
  \setlength{\itemsep}{0pt}\setlength{\parskip}{0pt}}\usepackage{longtable,booktabs,array}
\usepackage{calc} % for calculating minipage widths
% Correct order of tables after \paragraph or \subparagraph
\usepackage{etoolbox}
\makeatletter
\patchcmd\longtable{\par}{\if@noskipsec\mbox{}\fi\par}{}{}
\makeatother
% Allow footnotes in longtable head/foot
\IfFileExists{footnotehyper.sty}{\usepackage{footnotehyper}}{\usepackage{footnote}}
\makesavenoteenv{longtable}
\usepackage{graphicx}
\makeatletter
\def\maxwidth{\ifdim\Gin@nat@width>\linewidth\linewidth\else\Gin@nat@width\fi}
\def\maxheight{\ifdim\Gin@nat@height>\textheight\textheight\else\Gin@nat@height\fi}
\makeatother
% Scale images if necessary, so that they will not overflow the page
% margins by default, and it is still possible to overwrite the defaults
% using explicit options in \includegraphics[width, height, ...]{}
\setkeys{Gin}{width=\maxwidth,height=\maxheight,keepaspectratio}
% Set default figure placement to htbp
\makeatletter
\def\fps@figure{htbp}
\makeatother

\usepackage{fancyhdr} \pagestyle{fancy} \usepackage{lastpage}
\KOMAoption{captions}{tablesignature}
\makeatletter
\@ifpackageloaded{tcolorbox}{}{\usepackage[skins,breakable]{tcolorbox}}
\@ifpackageloaded{fontawesome5}{}{\usepackage{fontawesome5}}
\definecolor{quarto-callout-color}{HTML}{909090}
\definecolor{quarto-callout-note-color}{HTML}{0758E5}
\definecolor{quarto-callout-important-color}{HTML}{CC1914}
\definecolor{quarto-callout-warning-color}{HTML}{EB9113}
\definecolor{quarto-callout-tip-color}{HTML}{00A047}
\definecolor{quarto-callout-caution-color}{HTML}{FC5300}
\definecolor{quarto-callout-color-frame}{HTML}{acacac}
\definecolor{quarto-callout-note-color-frame}{HTML}{4582ec}
\definecolor{quarto-callout-important-color-frame}{HTML}{d9534f}
\definecolor{quarto-callout-warning-color-frame}{HTML}{f0ad4e}
\definecolor{quarto-callout-tip-color-frame}{HTML}{02b875}
\definecolor{quarto-callout-caution-color-frame}{HTML}{fd7e14}
\makeatother
\makeatletter
\makeatother
\makeatletter
\makeatother
\makeatletter
\@ifpackageloaded{caption}{}{\usepackage{caption}}
\AtBeginDocument{%
\ifdefined\contentsname
  \renewcommand*\contentsname{Table des matières}
\else
  \newcommand\contentsname{Table des matières}
\fi
\ifdefined\listfigurename
  \renewcommand*\listfigurename{Liste des Figures}
\else
  \newcommand\listfigurename{Liste des Figures}
\fi
\ifdefined\listtablename
  \renewcommand*\listtablename{Liste des Tables}
\else
  \newcommand\listtablename{Liste des Tables}
\fi
\ifdefined\figurename
  \renewcommand*\figurename{Figure}
\else
  \newcommand\figurename{Figure}
\fi
\ifdefined\tablename
  \renewcommand*\tablename{Tableau}
\else
  \newcommand\tablename{Tableau}
\fi
}
\@ifpackageloaded{float}{}{\usepackage{float}}
\floatstyle{ruled}
\@ifundefined{c@chapter}{\newfloat{codelisting}{h}{lop}}{\newfloat{codelisting}{h}{lop}[chapter]}
\floatname{codelisting}{Listing}
\newcommand*\listoflistings{\listof{codelisting}{Liste des Listings}}
\makeatother
\makeatletter
\@ifpackageloaded{caption}{}{\usepackage{caption}}
\@ifpackageloaded{subcaption}{}{\usepackage{subcaption}}
\makeatother
\makeatletter
\@ifpackageloaded{tcolorbox}{}{\usepackage[skins,breakable]{tcolorbox}}
\makeatother
\makeatletter
\@ifundefined{shadecolor}{\definecolor{shadecolor}{rgb}{.97, .97, .97}}
\makeatother
\makeatletter
\makeatother
\makeatletter
\makeatother
\makeatletter
\@ifpackageloaded{fontawesome5}{}{\usepackage{fontawesome5}}
\makeatother
\ifLuaTeX
\usepackage[bidi=basic]{babel}
\else
\usepackage[bidi=default]{babel}
\fi
\babelprovide[main,import]{french}
% get rid of language-specific shorthands (see #6817):
\let\LanguageShortHands\languageshorthands
\def\languageshorthands#1{}
\ifLuaTeX
  \usepackage{selnolig}  % disable illegal ligatures
\fi
\IfFileExists{bookmark.sty}{\usepackage{bookmark}}{\usepackage{hyperref}}
\IfFileExists{xurl.sty}{\usepackage{xurl}}{} % add URL line breaks if available
\urlstyle{same} % disable monospaced font for URLs
\hypersetup{
  pdftitle={Exercices - Algorithmes de tris},
  pdflang={fr},
  colorlinks=true,
  linkcolor={blue},
  filecolor={Maroon},
  citecolor={Blue},
  urlcolor={Blue},
  pdfcreator={LaTeX via pandoc}}

\title{Exercices - Algorithmes de tris}
\usepackage{etoolbox}
\makeatletter
\providecommand{\subtitle}[1]{% add subtitle to \maketitle
  \apptocmd{\@title}{\par {\large #1 \par}}{}{}
}
\makeatother
\subtitle{S6 - Algorithmique (1)}
\author{}
\date{}

\begin{document}
\maketitle
\lhead{Spécialité NSI} \rhead{Première} \chead{} \cfoot{} \lfoot{Lycée \'Emile Duclaux} \rfoot{Page \thepage/\pageref{LastPage}} \renewcommand{\headrulewidth}{0pt} \renewcommand{\footrulewidth}{0pt} \thispagestyle{fancy} \vspace{-2cm}

\ifdefined\Shaded\renewenvironment{Shaded}{\begin{tcolorbox}[borderline west={3pt}{0pt}{shadecolor}, frame hidden, enhanced, interior hidden, boxrule=0pt, breakable, sharp corners]}{\end{tcolorbox}}\fi

\emph{Les exercices précédés du symbole \faIcon{desktop} sont à faire
sur machine, en sauvegardant le fichier si nécessaire.}

\emph{Les exercices précédés du symbole \faIcon{pencil-alt} doivent être
résolus par écrit.}

\begin{tcolorbox}[enhanced jigsaw, title=\textcolor{quarto-callout-important-color}{\faExclamation}\hspace{0.5em}{Important}, rightrule=.15mm, colbacktitle=quarto-callout-important-color!10!white, left=2mm, leftrule=.75mm, toptitle=1mm, colback=white, breakable, coltitle=black, colframe=quarto-callout-important-color-frame, toprule=.15mm, bottomtitle=1mm, bottomrule=.15mm, arc=.35mm, titlerule=0mm, opacitybacktitle=0.6, opacityback=0]

Notebook Capytale pour les exercices 1 et 2 :
\href{https://capytale2.ac-paris.fr/web/c/136d-1599664}{Capytale}

\end{tcolorbox}

\hypertarget{fa-desktop-exercice-1}{%
\subsection{\texorpdfstring{\faIcon{desktop} Exercice
1}{ Exercice 1}}\label{fa-desktop-exercice-1}}

\textbf{Cet exercice est à faire dans Capytale.}

Écrire une fonction \texttt{trie\_chaine} qui prend en argument une
\textbf{liste} de chaînes de caractères et qui \textbf{modifie} cette
liste en la triant en fonction du nombre de lettres. Cette fonction ne
renvoie rien.

Tester la fonction avec la liste
\texttt{{[}"un",\ "deux",\ "trois",\ "quatre",\ "cinq",\ "six",\ "sept",\ "huit",\ "neuf",\ "dix"{]}}.

\hypertarget{fa-desktop-exercice-2-le-tri-uxe0-bulles}{%
\subsection{\texorpdfstring{\faIcon{desktop} Exercice 2 : le tri à
bulles}{ Exercice 2 : le tri à bulles}}\label{fa-desktop-exercice-2-le-tri-uxe0-bulles}}

\textbf{Cet exercice est à faire dans Capytale.}

L'algorithme de tri à bulles est le suivant :

\begin{itemize}
\tightlist
\item
  On parcourt la liste de gauche à droite.
\item
  Si deux éléments consécutifs sont dans le mauvais ordre, on les
  échange.
\item
  Si, à l'étape précédente, au moins un échange a eu lieu, on recommence
  à l'étape 1.
\item
  Sinon, la liste est triée et on arrête.
\end{itemize}

\begin{enumerate}
\def\labelenumi{\arabic{enumi}.}
\tightlist
\item
  Écrire toutes les étapes du tri à bulles pour la liste
  \texttt{{[}5,\ 3,\ 2,\ 4,\ 1{]}}.
\item
  Soit \(n\) un entier naturel non nul et \(L\) une liste de \(n\)
  entiers rangés dans l'ordre décroissant (pire des cas). Combien
  d'échanges sont nécessaires pour trier \(L\) dans l'ordre croissant ?
  En déduire une évaluation de la complexité de cet algorithme.
\item
  Écrire une fonction \texttt{tri\_bulles} qui prend en argument une
  \textbf{liste} de nombres et qui \textbf{modifie} cette liste en la
  triant par ordre croissant en utilisant l'algorithme du tri à bulles.
  Cette fonction ne renvoie rien.
\item
  Ajouter une variable \texttt{compteur} dans la fonction
  \texttt{tri\_bulles} qui compte le nombre d'échanges effectués. Ce
  nombre doit être renvoyé par la fonction. Tester la fonction avec la
  liste \texttt{{[}5,\ 3,\ 2,\ 4,\ 1{]}} et vérifier que le compteur
  vaut bien 6.
\end{enumerate}

\hypertarget{fa-desktop-t.p.-bilan-et-compluxe9ments}{%
\subsection{\texorpdfstring{\faIcon{desktop} T.P. : Bilan et
compléments}{ T.P. : Bilan et compléments}}\label{fa-desktop-t.p.-bilan-et-compluxe9ments}}

\begin{tcolorbox}[enhanced jigsaw, title=\textcolor{quarto-callout-important-color}{\faExclamation}\hspace{0.5em}{Important}, rightrule=.15mm, colbacktitle=quarto-callout-important-color!10!white, left=2mm, leftrule=.75mm, toptitle=1mm, colback=white, breakable, coltitle=black, colframe=quarto-callout-important-color-frame, toprule=.15mm, bottomtitle=1mm, bottomrule=.15mm, arc=.35mm, titlerule=0mm, opacitybacktitle=0.6, opacityback=0]

Notebook Capytale pour ce T.P. :
\href{https://capytale2.ac-paris.fr/web/c/a99d-1599680}{Capytale}

Ce T.P. est à faire dans Capytale en suivant le lien ci-dessus.

\end{tcolorbox}



\end{document}
