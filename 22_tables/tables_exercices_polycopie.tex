\documentclass[11pt,a4paper]{article}

\usepackage[utf8]{inputenc}
\usepackage[T1]{fontenc}
\usepackage[french]{babel}
\usepackage[a4paper,margin=2cm]{geometry}
\usepackage{lmodern}
\usepackage{microtype}
\usepackage{lastpage}
\usepackage{fancyhdr}
\usepackage{xcolor}
\usepackage{enumitem}
\usepackage{hyperref}
\usepackage{fontawesome5}
\usepackage[most]{tcolorbox}

\setlength{\parindent}{0pt}
\setlength{\parskip}{0.5em}
\everymath{\displaystyle}

\hypersetup{colorlinks=true,linkcolor=blue!50!black,urlcolor=blue!50!black}

\pagestyle{fancy}
\fancyhf{}
\fancyhead[L]{Première NSI}
\fancyhead[R]{Chapitre 5.3 -- Traitements de données en tables (Exercices)}
\fancyfoot[L]{Lycée Émile DUCLAUX}
\fancyfoot[R]{Page \thepage/\pageref{LastPage}}
\renewcommand{\headrulewidth}{0.4pt}
\renewcommand{\footrulewidth}{0.4pt}
\setlength{\headheight}{14pt}

\fancypagestyle{plain}{
  \fancyhf{}
  \fancyhead[L]{Première NSI}
  \fancyhead[R]{Chapitre 5.3 -- Traitements de données en tables (Exercices)}
  \fancyfoot[L]{Lycée Émile DUCLAUX}
  \fancyfoot[R]{Page \thepage/\pageref{LastPage}}
  \renewcommand{\headrulewidth}{0.4pt}
  \renewcommand{\footrulewidth}{0.4pt}
}

\newtcolorbox{importantbox}[1]{
  colback=red!3,
  colframe=red!60!black,
  title={\faExclamationTriangle\ \ #1},
  fonttitle=\bfseries
}

\newcommand{\zone}[1][2.4cm]{%
\par\noindent\fbox{\begin{minipage}[c][#1][c]{0.985\linewidth}\mbox{}\end{minipage}}\par\vspace{0.3em}}

\begin{document}

\begin{center}
{\LARGE \textbf{Exercices -- Traitements de données en tables}}\\[0.4em]
\end{center}

\textit{Les exercices précédés du symbole \faDesktop{} sont à faire sur machine.}

\textit{Les exercices précédés du symbole \faPen{} sont à résoudre par écrit.}

\begin{importantbox}{Notebook Capytale}
Pour les exercices 1 et 2 : \url{https://capytale2.ac-paris.fr/web/c/0681-1609936}
\end{importantbox}

\section*{\faDesktop{} Exercice 1}

On utilise le fichier CSV \texttt{educ\_cantal.csv} (encodage UTF-8).

\textbf{Avertissement :} certaines comparaisons portent sur des valeurs numériques, il faudra convertir les chaînes si nécessaire.

\begin{enumerate}[leftmargin=*]
  \item Importer ce fichier dans un programme Python et indexer les données sous la forme d'un tableau de dictionnaires nommé \texttt{table\_educ}.
  \item Combien d'enregistrements contient cette table ?
  \item Opérations de \textbf{sélection} :
  \begin{enumerate}
    \item En utilisant une boucle, définir le tableau \texttt{etab\_AURILLAC} contenant les établissements situés à Aurillac.
    \item En compréhension, obtenir la table des établissements privés du Cantal.
  \end{enumerate}
  \item Opérations de \textbf{projection} :
  \begin{enumerate}
    \item Avec une boucle, obtenir le tableau des codes UAI.
    \item En compréhension, obtenir le tableau des noms d'établissements.
    \item Obtenir le tableau des communes sans répétition.
  \end{enumerate}
  \item Opérations de \textbf{tri} :
  \begin{enumerate}
    \item Trier par ordre croissant de codes postaux.
    \item Trier d'ouest en est puis du nord au sud.
    \item Trier par commune puis, dans chaque commune, statut (public avant privé).
  \end{enumerate}
  \item Opération de \textbf{jointure} :
  \begin{enumerate}
    \item Importer \texttt{population\_Cantal.csv} dans un tableau de dictionnaires nommé \texttt{population}.
    \item Combien d'enregistrements contient cette table ?
    \item Effectuer la jointure entre \texttt{table\_educ} et \texttt{population} dans \texttt{new\_table} en ajoutant un champ \texttt{population}.
    \item Enregistrer \texttt{new\_table} dans \texttt{exo1\_jointure.csv}.
    \item Ouvrir le fichier dans un tableur, repérer les enregistrements sans population, expliquer la cause et proposer une correction.
  \end{enumerate}
\end{enumerate}

\begin{importantbox}{À retenir}
Lorsqu'on rapproche deux tables, il faut vérifier la cohérence des noms d'attributs, des formats de données (domaines de valeurs), des doublons, etc.

Un travail de \textbf{formatage} préalable des données est souvent nécessaire.
\end{importantbox}

\newpage

\section*{\faDesktop{} Exercice 2}

On reprend le fichier \texttt{population\_Cantal.csv}. Pour chaque question, écrire des instructions Python.

\begin{enumerate}[leftmargin=*]
  \item Construire la table triée par population décroissante.
  \item Construire le tableau des noms de communes dont le nom complet se termine par \og AC\fg{}. Combien y en a-t-il ?
  \item Construire la table des communes dont la population est comprise entre 1000 et 2000 habitants (bornes incluses).
  \item Question ouverte : construire le tableau des noms de communes composés de plusieurs mots.
\end{enumerate}

\end{document}
